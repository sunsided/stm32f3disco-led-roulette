\documentclass{standalone}
\usepackage{graphicx}
\usepackage{tikz}
\usepackage{tikz-3dplot}
\usetikzlibrary{backgrounds}

\usepackage{mathpazo}

\begin{document}

% Define the file path of your image
\newcommand{\imagepath}{reference/stm32f3discovery-top-500x500.jpg}

\begin{tikzpicture}
    % Assume fixed dimensions for the image
    \pgfmathsetmacro{\imagewidth}{10}
    \pgfmathsetmacro{\imageheight}{10}

    % Draw a white background
    \begin{scope}[on background layer]
        \fill[white] (-1.0,0) rectangle (\imagewidth,\imageheight);
    \end{scope}

    % Load the image
    \node[anchor=south west,inner sep=0] (image) at (0,0) {\includegraphics[width=10cm]{\imagepath}};

    % Define 3D coordinate system directly without extracting image dimensions
    \begin{scope}[x={(\imagewidth cm,0)}, y={(0,\imageheight cm)}, z={(0cm,0cm,1cm)}, canvas is xy plane at z=0]
        \draw[line width=4mm, red, -stealth] (0.5, 0.55) -- (0.9, 0.4) node[below] {\LARGE \textbf X};
        \draw[line width=1mm, purple, -stealth, ->] (0.89, 0.49) arc (65:235:0.9mm);

        \draw[line width=4mm, green, -stealth] (0.5, 0.55) -- (0.78, 0.9) node[right] {\LARGE \textbf Y};
        \draw[line width=1mm, olive, -stealth, <-] (0.78, 0.8) arc (-45:200:0.9mm);

        \draw[line width=4mm, blue, -stealth] (0.5, 0.55) -- (0.5, 1.0) node[right] {\LARGE \textbf Z};
        \draw[line width=1mm, cyan, -stealth, ->] (0.40, 0.9) arc (180:360:0.9mm);
    \end{scope}
\end{tikzpicture}

\end{document}